%%=========================================
\chapter{Theoretical Background}
\label{chp:theory}
\begin{info}
	Within this 'info' environment you can write stuff you don't want to be included in the final version of the report
\end{info}

In the theoretical background, I will start by introducing the different datasources that serves as input to Vake's pipline. A big part of this project is to understand the data, in order to know what to do with it later on.
Next up, I will explain how Vake uses this data to create a pipeline for ship detection. This will be useful for understanding how to scope my project later on. 
The last part of the theoretical background will contain some state-of-the-art modelling of AIS-predictions and reconstruction. This will help me to understand where the technology is today, and how I can apply general solutions to my specific challenge. 
%%-----------------------------------------
\section{Automatic Identification System (AIS)}
\begin{info}{}
	This is a simple section saved in a separate file.
\end{info}
%%-----------------------------------------
AIS is a large integrated system that combines equipment and protocols to send, recieve and decode navigational information from ships at sea. There are landmounted receivers, satelitte receivers and ship-mounted transcievers that constantly sends, receives and utilizes this information to improve maritime security and enhance monitoring, among other things.

According to the International Maritime Organization (IMO) \cite{MarineSafteyComittee}, the transcievers mounted on the individual ships must provide identity, ship-type, position, course, speed, and other safety-related information. 
The requlation enforced by IMO in 2002 also states that all ships of 300 gross tonnage or above engaging on international voyages, as well as all cargo ships above 500 gross tonnage and all passenger ships must be equipped with such technology. 

The global adaption of this technology has lead to an incredible amount of data with a large number of use-cases not imagined when IMO first enforced the system.  

\todo[inline]{List use cases for AIS, mabye find the source where IMO regretted public AIS-data}


\todo[inline]{Sub-chapters: How AIS works}
\todo[inline]{Sub-chapters: Protocol}
\todo[inline]{Sub-chapters: Weaknesses}


%%=========================================

\section{Data Management}
\begin{info}{}
	This is a simple section saved in a separate file.
\end{info}
%%-----------------------------------------



%%=========================================
\section{Sentinel-2 Images}
\begin{info}{}
	This is a simple section saved in a separate file.
\end{info}
%%-----------------------------------------



%%=========================================

\section{Vake and Vake's pipeline}
\begin{info}{}
	This is a simple section saved in a separate file.
\end{info}
%%-----------------------------------------




%%=========================================

%%=========================================
\section{Current research on AIS-preprocessing}
\begin{info}{}
	This is a simple section saved in a separate file.
\end{info}
%%-----------------------------------------




%%=========================================

%%=========================================
\section{Current research on AIS-predictions and -reconstruction}
\begin{info}{}
	This is a simple section saved in a separate file.
\end{info}
%%-----------------------------------------





%%=========================================


%%=========================================