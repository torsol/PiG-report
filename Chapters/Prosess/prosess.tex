\chapter{Applikasjon}

Dette kapittelet tar for seg arbeidsgangen i prosjektet, en detaljert arkitekturskisse, og en gjennomgang av all funksjonalitet knyttet til geoprosessering, filbehandling, grensesnitt og brukerveiledning.

\section{Arbeidsgang}

Det første som ble gjort i forbindelse med applikasjonen var å få en forståelse av hva som har blitt gjort tidligere, med tanke på valg av grensesnitt og funksjonalitet. Dette var også med på å få en forståelse av hva som var forventet og hva som ligger i ordene \textit{viktig GIS-funksjonalitet}. Et dokument som inneholdt geoprosseseringsfunksjonalitet og lagbehandlingsfunksjoner ble laget utifra tilgjengelige prosjekter\cite{Johanessen}\cite{Strand}\cite{Eglaaen}\cite{Villanger}\cite{Jakobsen} som lå tilgengelig på nett.

De ulike geoprosesseringsoperasjonene inkluderte blant annet:

\begin{frame}

    \begin{tabular}{p{0.4\textwidth}p{0.5\textwidth}}

        \begin{itemize}
            \item Clip
            \item Buffer
            \item Union
            \item Dissolve
        \end{itemize} &

        \begin{itemize}
            \item Difference
            \item Intersection
            \item Bounding box
            \item Filtering
        \end{itemize}
    \end{tabular}

\end{frame}

Mens typiske lagbehandlingsfunksjoner var:

\begin{frame}

    \begin{tabular}{p{0.4\textwidth}p{0.5\textwidth}}

        \begin{itemize}
            \item Legge til lag
            \item Fjerne lag
            \item Endre navn på lag
            \item Endre farge på navn
        \end{itemize} &

        \begin{itemize}
            \item Vise/skjule lag
            \item Zoome til lag
            \item Laste ned lag
        \end{itemize}
    \end{tabular}

\end{frame}

Etter å ha sjekket at de programmeringsspråkene jeg ønsket å jobbe med hadde støtte for denne typen operasjoner var neste steg å sette opp klient-prosjektet og server-prosjektet.  


- Repositories

- Få opp kart og sidebar i frontend

- Få opp og kommunisere med backend

- Proof of concept der client kommuniserer med server

- Ekstern hosting

- Implementering av funksjonalitet

\section{Detaljert arkitektur}

\section{Funksonalitet}

\subsection{Brukergrensesnitt}

- Tilbakemelding på feil eller suksess

- Design

- Minst mulig sidemenyer

- Kartdesign

\subsection{Geoprosessering}

\subsection{Lagbehandling}

\subsection{Brukerveiledning}

\section{Hosting}
