\chapter[Feature Engineering]{Feature Engineering}
\label{chp:labelkey}
\begin{info}
	% Within this 'info' environment you can write stuff you don't want to be included in the final version of the report
\end{info}

In this section, I will be explaining what feature engineering is, give my motivation for doing feature engineering and my methodology for creating said features. 
%%-----------------------------------------
\section{What is feature engineering}
\begin{info}{}
	Tilhører kanskje teoridelen, men det kan jeg fikse senere. 
\end{info}

Features are individual and measurable properties of a characteristic being observed. Choosing and generating good features are an important part of machine learning. These are datadriven methods where the significance and quality of the training data is crucial for the success of the classification or regression problems these models usually are applied. 

Feature engineering is the dicipline of creating new features, based on already existing data or external sources. 

%%-----------------------------------------
\section{Why is feature engineering important?}
\begin{info}{}
\end{info}

%%-----------------------------------------
\section{Static Ship Characteristic Features}
\begin{info}{}
\end{info}

- Likelihood of porting in tile


%%-----------------------------------------



%%-----------------------------------------
\section{Kinematic Ship features}
\begin{info}{}
\end{info}



%%-----------------------------------------

%%-----------------------------------------
\section{Spatial Awareness Features}
\begin{info}{}
\end{info}

- Distance to land

- depth/grunne/skjær

- Density Map

- Distance to port

\section{Temporal Awareness Features}
\begin{info}{}
\end{info}

- Time of day

- Degree of light


%%-----------------------------------------