\chapter{Introduksjon}
Introduksjonen gir først en grundig innføring i valgt oppgave og motivasjonen bak. Videre defineres noen overordnede formål med oppgaven som har vært drivende gjennom utvikling og generell prosess. Tilslutt blir det valgte rammeverket som oppgaven bygger på presentert.

\section{Bakgrunn for oppgave}
I faget TBA4251 - Programmering i Geomatikk skal man velge og utføre en av tre programmeringsprosjekter i et ledet selvstudium. Det er prosjektet, samt den tilhørende rapporten som utgjør vurderingen i faget, til en verdi av 7.5 studiepoeng. Dette prosjektet ble utførst høsten 2020, som en del av studieløpet innenfor Ingeniørvitenskap \& IKT, med fordypning innenfor geomatikk. 

Valget av prosjekt falt på GIS-oppgaven, til fordel for 3D-modellerings og Geodesi. GIS står for Geographic Information System, og oppgaven går ut på å lage et slikt system som har implementert grunnleggende GIS-funksjonalitet \cite{midtbøa}. Som en del av prosjektoppgaven min måtte jeg sette meg inn i nye romlige programmeringsbiblioteker for å manipulere posisjonsdata, så da falt det naturlig å velge en oppgave i programmering i geomatikk som også krevde implementering av geoprosesseringsverktøy. På den måten kunne jeg dra nytte av ny kunnskap innenfor begge emner, som styrket både prosjektoppgaven og denne GIS-oppgaven. 

\section{Formål med applikasjonen}

Formålene med applikasjonen er det som burde drive foksuet på utviklingen. I oppgaveinstruksen som er gitt til prosjektet \cite{midtbøa} fremkommer det flere praktiske gjøremål i tillegg til det overordnede målet om å "bruke kunnskap om datateknologi sammen med kunnskap om geografiske informasjonssystemer. Essensen av de praktiske gjøremålene er som følger: 

\begin{enumerate}
    \item applikasjonen må inneholde viktig GIS-funksjonalitet
    \item applikasjonen må kunne ta inn og analysere en viss mengde data
    \item applikasjonen bør være brukevennlig
    \item applikasjonen bør ha et veldesignet brukergrensesnitt
\end{enumerate}

Videre ble det spesifisert noen ekstra gjøremål i retningslinjene \cite{midtbø}. 

\begin{enumerate}[resume]
    \item Programkode skal være godt dokumentert
    \item Er applikasjonen web-basert må den installeres på et permanent sted som ikke forsvinner med studentbruker
\end{enumerate}

Disse gjøremålene blir toneangivende for utviklingen, men jeg har også valgt å inkludere et par egendefinerte fokusområder. Dette er mest gjøremål for å maksimere brukervennligheten, og for å skille seg ut fra tidligere oppgaver. 
\begin{enumerate}[resume]
    \item Applikasjonen bør utformes for å ha mest mulig interaksjon på selve kartet (ikke i sidemenyer)
    \item En brukerveiledning som er inkludert i selve applikasjonen
    \item Applikasjonen bør benytte seg av en arkitektur som er modulerbar
\end{enumerate}

\section{Bakgrunn for valg av løsningsarkitektur}
Valg av løsningsarkitektur var i stor grad inspirert av to faktorer. Den første var erfaringen med tidligere prosjekter der det ble brukt webgrensesnitt for å visualisere og manipulere kart. Det finnes flere gode kart-biblioteker for web, i tillegg til mange etablerte rammeverk for å holde styr på tilstand, minne og visuell utforming. 

Den andre faktoren var ønsket om å bruke noe av den samme teknologien som var benyttet i prosjektoppgaven. Dette er python-baserte rammeverk som er egnet for store data og rask geoprosessering. For å best benytte seg av dette ble det tydelig at en frontend basert på javascriptbiblioteker med hovedoppgave 

\section{Bakgrunn for valg av programmeringsspråk}
For å kunne velge programmeringsspråk 




\cite{Johanessen}