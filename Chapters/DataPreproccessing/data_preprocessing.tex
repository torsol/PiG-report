%%=========================================
\chapter[Data Preprocessing]{Data Preprocessing}
\begin{info}
	Snakke om hvorfor valget falt på MGRS tile 32VLQ. Snakke om hvor mye data jeg skal se på i første omgang. Snakke om det å sample det ned med henhold til tid, posisjon, grid KPFE
\end{info}
%%-----------------------------------------

The amount of AIS data available for analysis is simply too dense and large to make any meaningful analysis within reasonable time. Therefore, I have to perform some kind of scoping in order to lower the wast amount. Breaking it down, this challenge consists of two tasks, picking an intelligent area of interest, and downsample the remaining data while retaining as many essential details as possible. This also constrains my task of AIS-reconstruction to a local area, this opens up the possibility for area-specific features.  


\section{Selection of Area}
\begin{info}{}
	This is a simple section saved in a separate file.
\end{info}
%%-----------------------------------------
The selection of area 
\todo[]{sdfsdf}

The previously mentioned MGRS-tiles that matches with the image products from the Sentinel-2 Mission drives the selection of area of study for this specialization project. The overlap could power region-based predictions as the areas themselves remains stationary and consistent over each retake of images.  




%%=========================================


\section{Downsampling}
\begin{info}{}
	This is a simple section saved in a separate file.
\end{info}
%%-----------------------------------------






%%=========================================

\section{Density map of ships}
\begin{info}{}
	This is a simple section saved in a separate file.
\end{info}
%%-----------------------------------------




%%=========================================

%%=========================================