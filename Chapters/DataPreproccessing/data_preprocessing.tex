%%=========================================
\chapter[Data Preprocessing]{Data Preprocessing}
\begin{info}
	Snakke om hvorfor valget falt på MGRS tile 32VLQ. Snakke om hvor mye data jeg skal se på i første omgang. Snakke om det å sample det ned med henhold til tid, posisjon, grid KPFE
\end{info}
%%-----------------------------------------

The amount of AIS data available for analysis is simply too dense and large to make any meaningful analysis within reasonable time. Therefore, I have to perform some kind of scoping in order to lower the wast amount. Breaking it down, this challenge consists of two tasks, picking an intelligent area of interest, and downsample the remaining data while retaining as many essential details as possible. This also constrains my task of AIS-reconstruction to a local area, this opens up the possibility for area-specific features.  


\section{Selection of Area}
%%-----------------------------------------
There are a lot of questions surrounding the selection of area. The area has to  be large enough to include enough AIS-data for meaningful analysis. Also, it can't be too large because the querying time will be too large for this exploratory part of the project. 

With spatial properties being a characteristic of AIS-data, some meaning could be extracted from the relations between points, both on an ellipsoid, but also on a projected plane. Therefore, we need to find a sensible area that that powers a simple and accurate projection. 
\todo[]{Get a reference from Vakes MGRS-density grid.}

The spatial properties of the VAKE-detections are detections divided using their respective MGRS-tiles. Therefore, for our selection of area, we need to fuse the overlapping MGRS-tiles for our selected area to get all the detections in that area. This is not a trivial task, as the tiles have overlap, and therefore multiple detections of the same ships in them. 

With this in mind, we need an area with a simple and accurate projection, and a way of relating it to the products of the VAKE-pipeline, which is given in overlapping MGRS-tiles. If we simply choose our area of interest to be such an MGRS-tile we acheive the first goal, and the second goal becomes trivial in execution, as there would be no overlap. 

We have acheived the property of an easy an intuitive projection, as the UTM-projection matches the MGRS-tiles. For the relatively small MGRS-tiles for 100km x 100km, the distrortions of the area (because UTM is a conformal projection) will hopefully prove negligible. What remains is to choose which MGRS-tile we will go for. 

In \cite{Tofting2018} they introduce what they call is a "ship likelihood constant" over all MGRS-tiles based on 20 days of global AIS-data. Based on their data, it can be confirmed that the norwegian coastline has a high density of registered AIS-messages. This empowers the assumption that our coastline has excellent AIS-coverage. Their density map can be explored at \cite{Tofting}

 \todo{Why that exact tile? Combination of land and water, can potentionally utilize more features}
\todo{The check that our assumptions is correct}



\section{Downsampling}
\begin{info}{}
	This is a simple section saved in a separate file.
\end{info}
%%-----------------------------------------






%%=========================================

\section{Density map of ships}
\begin{info}{}
	This is a simple section saved in a separate file.
\end{info}
%%-----------------------------------------




%%=========================================

%%=========================================