%%=========================================
\chapter{Introduction}
\label{chp:introduction}
\begin{info}
	The first chapter of a well-structured thesis is always an introduction, setting the scene with background, problem description, objectives, limitations, and then looking ahead to summarize what is in the rest of the report. This is the part that readers look at first---\emph{so make sure it hooks them!} Problem formulation, Related work, What remains to be done
\end{info}
%%-----------------------------------------
The first chapter will give an introduction to the problem and the motivation behind endavouring on the given task. Next, I will summarize the main objectives for the project, the approach for acheiving these. Finally the chapter will include a brief summary of the limitations for the project and the outline for the following chapters. 

%%=========================================
\section{Background}
\label{sec:background}
\begin{info}
	In this section, you should present the problem that you are going to investigate or analyse; why this problem is of interest; what has, so far, been done to solve the problem, and which parts of the problem that remain. Should include \textbf{Problem Formulation}, \textbf{Related Work}, \textbf{What Remains to be Done?}.
\end{info}
%%-----------------------------------------

Every year, the different course coordinators at Engineering \& ICT holds a presentation in order to motivate the lower grade students to choose their specialization. As a part of the geomatics presentation in the spring of 2018, I clearly remember Adrian Tofting and Lars-Henrik Berg-Jensen pitching their master thesis work. They were a part of opening my eyes to geomatics, and to realise the possibilites of combining it with cutting-edge computer science. Their master thesis consisted of automatic labeling of ships on satelitte images, creating a dataset for powering automatic ship detection using machine learning. Since then, they have used their master thesis as a springboard for founding the startup VAKE, with a mission of acheiving global real-time detection of vessels. 

\begin{info}
	What am I doing
\end{info}

I am honored to be able to work with VAKE, and to do my best efforts to do a meaningful contribution to their mission. In this paper, I will conduct exploratory work on a particular part of their pipeline, namely the part that handles the matching of detections with the identity of the ship. The ambition for the following master thesis is to use this work for creating an improved model to increase matching accuracy. 

\begin{info}
	Why is this interesting (Motivation)
\end{info}
In general, increased maritime surveillance is benificial for both the private and public sectors on a national and global scale. With the ability to monitor all ships independent of the navigational equipment onboard the authorities can register and respond to a multitude of challenges at sea. This includes, but is not limited to, the tampering of navigational equipment, illegal fishing, monitoring of foreign military activity, illegal transportation of goods, illegal transhipment of cargo at sea, aiding in the tracking of ships lost at sea and the detection of abnormal traffic and deviance from the norm. If illegal activity is deterred, and search- and rescue-missions are aided, this will lead to a safer and fairer sea space for all. Detection of ships on satelitte images has been acheived, but the detections themselves offers low value if they can't be put in context of the actual ship identities. By improving the matching, tying the detections to the identification of the ships, this will create actionable data for the affected authorities.     

\begin{info}
	What has, so far, been done to solve the problem
\end{info}
VAKE already has implemented a method for matching their detections to the ship identity by using data from the Automatic Identifiaction System (AIS). By interpolating between the last known position before the image and the first known position after the image, they employ a linear interpolation to estimate the ship position at the time of the image. However, this method has flaws, and can't account for large time periods and/or advanced ship motion. This approach does not capture nearly as many ships as more advances trajectory modeling hopefully can do.

\begin{info}
	What remains?
\end{info}
There is a lot that remains before a new matching model can be developed and tested. We need to gain insight in what kind of data is available to aid the matching, what state-of-the-art models exists, and we need to get an overview of exisiting research on the topic. 





%%=========================================
\section{Objectives}
\label{sec:objectives}
\begin{info}
	The objectives shall be written as \emph{fundamental objectives} telling what to do and not \emph{means objectives} telling how to do it.

	All objectives shall be stated such that we, after having read the thesis, can see whether or not you have met the objective. ``To become familiar with \ldots'' is therefore not a suitable objective.
\end{info}
%%-----------------------------------------
In order to create the best foundation for my master thesis, I need an in-depth knowledge of the data sources I have available, what state-of-the-art research has been conducted on the domain, and what tools that will give me the best possibility to succeed with my work. I have therefore formulated the following goals aiming to achieve this:
\begin{enumerate}
	\item Summarize what AIS data is
	\item Summarize what AIS data is available for this project
	\item Summarize what other data is available from Vake
	\item Summarize the current implementation at Vake and identify key takeaways
	\item Summarize what tools is available for big data analysis
\end{enumerate}

Next up is establishing knowledge of the state-of-the-art methods utilized to handle AIS-data and to use AIS-data in predictions
\begin{enumerate}
	\item Summarize preprocessing done on AIS-data and why it could be necessary
	\item Summarize different models of predicitions on AIS-data
\end{enumerate}

Lastly, I will conduct preprocessing and feature engineering on the data in order to establish a framework to expand upon in the master thesis
\begin{enumerate}
	\item Explain why preprocessing could be helpful
	\item List and show relevant preprocessing reducing the amount of data
	\item Explain the process of feature engineering
	\item List and show relevant feature extraction from the available AIS-data
\end{enumerate}


%%========================================{}=
\section{Approach}
\label{sec:approach}
\begin{info}
	Here you should describe the (scientific) approach and experiments that you will use or have used to solve the problem and meet your objectives and tasks. Experiments may in this context relate analyses you need to carry out in order to investigate a specific hypothesis, task objective, or similar. You should specify the approach and experiments for each objective and/or task. It is preferred that you supplement your explanation of the approach with an illustration.

	If there are any ethical problems related to your approach, these should be highlighted and discussed.
\end{info}
%%-----------------------------------------
This project is a litterature study, as well as a preliminiary exploration of the existing methods and data that I have to work with for the given modeling task. I will carry out both objectives simultaneously, having a goal of letting the exploration of each objective affect the other. 

The litterature review is carried out by  

For the feature engineering, I have tried to adhere to the given principles of good features. However, an important part of feature extraction and feature engineering is the verification of the feature importance, but since I do not have a model yet that is not possible. However, I will strive to give a quantitative and qualitative judgement over the features found. 




%%=========================================
\section{Contributions}
\label{sec:contributions}
\begin{info}
	Here you give a list of your main contributions in the project or master work.
\end{info}
%%-----------------------------------------
\begin{enumerate}
	\item A review of research relevant to the problem
	\item A short analysis of the weaknesses of the current implementation
	\item A benchmarking of the current AIS-matching
	\item New pipeline for processing AIS-data for future predictions
	\item New recipies for feature extraction of AIS messages
\end{enumerate}

%%=========================================
\section{Limitations}
\label{sec:limitations}
\begin{info}
	In this section you describe the limitations of your study. These may be related to the study object (physical limitations, operational limitations), to the environmental and operational conditions, to the thoroughness of the analysis, and so on.
\end{info}
%%-----------------------------------------
The limitations of this pre-study has to do with scope and how the data is processed. An ideal solution to this problem is a generalized algorithm that works in all timezones, all over the earth and that can make advanced predictions using all data available at near real time. 

For this pre-study I have focused on one \cite{Tofting2018} particular MGRS-tile, with a small subset of the total data, using features that is powered by local data. A prediction model developed on this data will probably not be able to generalize well for other tiles than the selected one. 

%%=========================================
\section{Outline}
\label{sec:outline}
\begin{info}
	Here, you give an overview of how the remaining part of the report is organized.
\end{info}
%%-----------------------------------------

\begin{itemize}
	\item Chapter 2. Theoretical background: 
	\item Chapter 3..N-2: The naming of the following chapters relies entirely on the specific topic in question. Proposed structure should be discussed with supervisor.
	\item Chapter N-1 Results: This chapter should be the last chapter \textit{before} ``Conclusions, discussion, and ideas for further work''
	\item Chapter N. Conclusions, discussion, and ideas for further work.
	\item Bibliography
	\item Appendix A etc. (as needed): Appendix A may for example be acronyms as shown here.
\end{itemize}

%%=========================================